\section{Метрические и топологические пространства}

\hfill \emph{Лекция от 16.09.2020}

Введём начальную терминологию, которой мы будем пользоваться на протяжении всего курса. Несмотря на то, что третьекурсники уже
сталкивались с частью ниже изложенных понятий, необходимо унифицировать используемый язык.

Начнём с метрического случая.

\begin{Def}
    Метрическое пространство -- пара $\left( X, \rho \right)$, где $X$ -- векторное пространство над полем скаляров $\K$, а 
    $\rho$ -- метрика на нём.
\end{Def}

\begin{Def}
    Метрика -- функция $\rho: X \longrightarrow \R$, удовлетворяющая трём условиям для любых $x, y, z \in X$:

    \begin{enumerate}
        \item $\rho(x, y) \geq 0$, $\rho(x, y) = 0 \iff x = y$
        \item $\rho(x, y) = \rho(y, x)$
        \item $\rho(x, y) \leq \rho(x, z) + \rho(z, y)$.s
    \end{enumerate}
\end{Def}

$\K$ будет обозначать либо $\R$, либо $\Cp$. На выбор конкретного поля скаляров будет обращено особое внимание в тех местах, где этот
 будет существенен.

\begin{Def}
    Топологическое пространство -- пара $(X, \tau)$. $X$ -- некоторое множество, а $\tau \subset 2^X$ -- топология на $X$, то есть система множеств, удовлетворяющая условиям:

    \begin{enumerate}
        \item $X, O \in \tau$
        \item $\bigcup\limits_{\alpha \in A} G_{\alpha} \in \tau$, если $\forall \alpha \in A: \, G_{\alpha} \in \tau$
        \item $\bigcap\limits_{k=1}^{N} G_k \in \tau$, если все $G_k \in \tau$.
    \end{enumerate}
\end{Def}

В последующем и топологические, и метрические пространства будут часто обозначаться только первым символом множества из пары, если для неё из контекста повествования очевидна
 топология или метрика. Далее по тексту МП -- метрическое пространство, ТП -- топологическое пространство.

До введения следующих необходимых определений приведём сводную таблицу понятий-аналогов из метрического и топологического случаев.

\begin{center}
\begin{tabular}{ |c|c| }
    Подпространство & Подпространство \\
    Ограниченное множество & - \\
    Расстояние м/у множествами & - \\
    Замыкание множества & + \\
    Замкнутые множество & + \\
    Внутренная точка & + \\
    Открытое ядро & + \\
    Открытое множество & + \\

    
\end{tabular}
\end{center}

\begin{Def}
    Подпространство метрического пространства $Y$ -- пара $(X, \rho)$, где $X \subset Y$, а $\rho$ -- метрика на $Y$.
\end{Def}

Метрику из подпространства называют индуцированной метрическим пространством $Y$.

\begin{Def}
    Индуцированная топологическим пространством $\left( Y, \tau_Y \right)$ топология для $X \subset Y$ -- $\tau_X = \{ G \cap X :\, G \in \tau_Y \}$.
    Пара $(Y, \tau_Y)$ называется подпространством $\left(X, \tau_X \right)$.
\end{Def}

\begin{Def}
    Диаметром множества $X$ в \MS называют $d = \sup\limits_{x, y \in X} \rho(a, y)$.
\end{Def}

\begin{Def}
    Множество $X$ в \MS называют ограниченным, если его диаметр меньше бесконечности.
\end{Def}

\begin{Def}
    Расстоянием между множествами $A$ и $B$ в \MS называют $\rho(A, B) = \inf\limits_{\substack{a \in A\\ b \in B}} \rho(a, b)$.
\end{Def}

\begin{Def}
    Множество $B(x, r) = \{ y \, | \, \rho(y, x) < r \}$ в метрическом пространстве при $r > 0$ называется открытым шаром.
\end{Def}
    Также для шаров могут встречаться обозначения $B_r(x)$ и $B(x)$.

\begin{Def}
    Множество $\overline{B}(x, r) = \{ y \, | \, \rho(y, x) < r \}$ в метрическом пространстве при $r > 0$ называется замкнутым шаром.
\end{Def}

\begin{Def}
    Элемент $x$ из \MS $X$ для множества $M \subset X$ называется точкой прикосновения, если $\rho(x, M) = 0$.
\end{Def}

\begin{Def}
    Элемент $x$ из \TS $X$ для множества $M \subset X$ называется точкой прикосновения, если $\forall B(x) \cap M \neq \emptyset$.
\end{Def}

Все точки прикосновения $x$ множества $M$ можно разделить на два вида:
\begin{enumerate}
    \item Предельные точки, т.е. $\forall B(x) \exists m \in M \, m \in B(x), m \neq x$;
    \item Изолированные точки множества $M$.
\end{enumerate}

\begin{Def}
    Замыканием множества называют его объединение с множеством точек прикосновения.
\end{Def}

\begin{Def}
    Множество $M$ называют замкнутым, если $M = \overline{M}$.
\end{Def}

В старых работах можно встретить обозначение замыкания множества через квадратные скобки: $[M]$. В этом курсе так будет обозначаться линейная оболочка.

\begin{Def}
    Точка $m \in X$ называют внутренней для множества $X$, если $\exists B(x, r) \, : \, B(x, r) \subset X$.
\end{Def}

\begin{Def}
    Открытым ядром множества $X$ называют множество его открытыъ точек. Обозначения: $\Int M$, $\mathring{M}$.
\end{Def}

\begin{Def}
    Множество $M$ называется открытым, если $\Int M = M$.
\end{Def}

\begin{Def}
    Множество $A$ называют плотным в $B$, если $B \subset \overline{A}$.
\end{Def}

\begin{Def}
    Множество $A$ называют всюду плотным в $B$, если $B = \overline{A}$.
\end{Def}

\begin{Def}
    Множество $A$ называют нигде не плотным в $B$, если $A$ не плотно ни в одном шаре из $B$ (или же $\overline{A}$ не содержит ни одного шара).
\end{Def}