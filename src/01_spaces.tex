\section{Метрические и топологические пространства}

\hfill \emph{Лекция от 16.09.2020}

Введём начальную терминологию, которой мы будем пользоваться на протяжении всего курса. Несмотря на то, что третьекурсники уже
сталкивались с частью ниже изложенных понятий, необходимо унифицировать используемый язык.

Начнём с метрического случая.

\begin{Def}
    Метрическое пространство -- пара $\left( X, \rho \right)$, где $X$ -- векторное пространство над полем скаляров $\K$, а 
    $\rho$ -- метрика на нём.
\end{Def}

\begin{Def}
    Метрика -- функция $\rho: X \longrightarrow \R$, удовлетворяющая трём условиям для любых $x, y, z \in X$:

    \begin{enumerate}
        \item $\rho(x, y) \geq 0$, $\rho(x, y) = 0 \iff x = y$
        \item $\rho(x, y) = \rho(y, x)$
        \item $\rho(x, y) \leq \rho(x, z) + \rho(z, y)$.s
    \end{enumerate}
\end{Def}

$\K$ будет обозначать либо $\R$, либо $\Cp$. На выбор конкретного поля скаляров будет обращено особое внимание в тех местах, где этот
 будет существенен.

\begin{Def}
    Топологическое пространство -- пара $(X, \tau)$. $X$ -- некоторое множество, а $\tau \subset 2^X$ -- топология на $X$, то есть система множеств, удовлетворяющая условиям:

    \begin{enumerate}
        \item $X, \emptyset \in \tau$
        \item $\bigcup\limits_{\alpha \in A} G_{\alpha} \in \tau$, если $\forall \alpha \in A: \, G_{\alpha} \in \tau$
        \item $\bigcap\limits_{k=1}^{N} G_k \in \tau$, если все $G_k \in \tau$.
    \end{enumerate}
\end{Def}

В последующем и топологические, и метрические пространства будут часто обозначаться только первым символом множества из пары, если для неё из контекста повествования очевидна
 топология или метрика. Далее по тексту МП -- метрическое пространство, ТП -- топологическое пространство.

До введения следующих необходимых определений приведём сводную таблицу понятий-аналогов из метрического и топологического случаев.

\begin{center}
\begin{tabular}{ |c|c| }
    \hline
    \bf{МП} & \bf{ТП} \\
    \hline
    Подпространство & + \\
    Ограниченное множество & - \\
    Расстояние м/у множествами & - \\
    Замыкание множества & + \\
    Замкнутые множество & + \\
    Внутренная точка & + \\
    Открытое ядро & + \\
    Открытое множество & + \\
    Сходящаяся последовательсноть & +, но не всегда \\
    \hline
    
\end{tabular}
\end{center}

\begin{Def}
    Подпространство метрического пространства $Y$ -- пара $(X, \rho)$, где $X \subset Y$, а $\rho$ -- метрика на $Y$.
\end{Def}

Метрику из подпространства называют индуцированной метрическим пространством $Y$.

\begin{Def}
    Индуцированная топологическим пространством $\left( Y, \tau_Y \right)$ топология для $X \subset Y$ -- $\tau_X = \{ G \cap X :\, G \in \tau_Y \}$.
    Пара $(Y, \tau_Y)$ называется подпространством $\left(X, \tau_X \right)$.
\end{Def}

\begin{Def}
    Диаметром множества $X$ в \MS называют $d = \sup\limits_{x, y \in X} \rho(a, y)$.
\end{Def}

\begin{Def}
    Множество $X$ в \MS называют ограниченным, если его диаметр меньше бесконечности.
\end{Def}

\begin{Def}
    Расстоянием между множествами $A$ и $B$ в \MS называют $\rho(A, B) = \inf\limits_{\substack{a \in A\\ b \in B}} \rho(a, b)$.
\end{Def}

\begin{Def}
    Множество $B(x, r) = \{ y \, | \, \rho(y, x) < r \}$ в метрическом пространстве при $r > 0$ называется открытым шаром.
\end{Def}
    Также для шаров могут встречаться обозначения $B_r(x)$ и $B(x)$.

\begin{Def}
    Множество $\overline{B}(x, r) = \{ y \, | \, \rho(y, x) < r \}$ в метрическом пространстве при $r > 0$ называется замкнутым шаром.
\end{Def}

\begin{Def}
    Элемент $x$ из \MS $X$ для множества $M \subset X$ называется точкой прикосновения, если $\rho(x, M) = 0$.
\end{Def}

\begin{Def}
    Элемент $x$ из \TS $X$ для множества $M \subset X$ называется точкой прикосновения, если $\forall B(x) \cap M \neq \emptyset$.
\end{Def}

\vspace{0.5cm}

Все точки прикосновения $x$ множества $M$ можно разделить на два вида:
\begin{enumerate}
    \item Предельные точки, т.е. $\forall B(x) \exists m \in M \, m \in B(x), m \neq x$;
    \item Изолированные точки множества $M$.
\end{enumerate}

\begin{Def}
    Замыканием множества называют его объединение с множеством точек прикосновения.
\end{Def}

\begin{Def}
    Множество $M$ называют замкнутым, если $M = \overline{M}$.
\end{Def}

В старых работах можно встретить обозначение замыкания множества через квадратные скобки: $[M]$. В этом курсе так будет обозначаться линейная оболочка.

\begin{Def}
    Точка $m \in X$ называют внутренней для множества $X$, если $\exists B(x, r) \, : \, B(x, r) \subset X$.
\end{Def}

\begin{Def}
    Открытым ядром множества $X$ называют множество его открытыъ точек. Обозначения: $\Int M$, $\mathring{M}$.
\end{Def}

\begin{Def}
    Множество $M$ называется открытым, если $\Int M = M$.
\end{Def}

\begin{Def}
    Множество $A$ называют плотным в $B$, если $B \subset \overline{A}$.
\end{Def}

\begin{Def}
    Множество $A$ называют всюду плотным в $B$, если $B = \overline{A}$.
\end{Def}

\begin{Def}
    Множество $A$ называют нигде не плотным в $B$, если $A$ не плотно ни в одном шаре из $B$ (или же $\overline{A}$ не содержит ни одного шара).
\end{Def}

\begin{Def}
    Последовательность элементов $\{x_n\}$ из \MS $M$ называется сходящейся к элементу $x_0 \in M$, если
    $\rho(x_n, x_0) \rightarrow 0$.
\end{Def}

\begin{Def}
    Пространство называется сепарабельным, если в нём существует счётное всюду плотное множество.
\end{Def}

\begin{Exercise}
    Докажите эквивалентность утверждений:
    \begin{itemize}
        \item $x$ -- точка прикосновения $M$
        \item $\exists \{m_k\} \subset M \, : \, m_k \rightarrow x$
    \end{itemize}
\end{Exercise}

\begin{Exercise}
    Докажите, что $C[a, b]$ сепарабельно. Метрика: $\rho(f, g) = \max\limits_{x} (f(x)-g(x))$. Указание: рассмотрите множество полиномов с рациональными коэффициентами.
\end{Exercise}

\vspace{0.5cm}
Сформулируем первые три теоремы.
\vspace{0.5cm}

\addcontentsline{toc}{subsection}{Теорема 1.1}
\begin{Theorem} \label{thm:thm_1_1}
$(X, \rho)$ -- \MS. $F \subset X$ -- замкнутое множество $\Longleftrightarrow X \setminus F$ -- открытое множество.
\end{Theorem}
\begin{Proof}
    Выберем любую точку $x \in X \setminus F$. Если удастся окружить её открытой окрестностью, не имеющей пересечений с $F$, то теорема будет доказана. Раз $x \not\in F$, то $x$ -- не точка прикосновения $F$.
    Значит, $\exists B(x) \cap F = \emptyset$ и $B(x) \subset X \setminus F$.
\end{Proof}

\vspace{0.5cm}

Со времён <<Теории множеств>> Хаусдорфа (немецкий математик, основатель топологии, 1868-1942) за открытыми множествами закрепилось обозначение $G$, за замкнутыми -- $F$.

Открытое ядро $M$ -- это наибольшее открытое множество в $M$, и потому оно может быть записано в виде $\bigcup\limits_{G \subset M} G$.

Замыкание $M$ -- наименьшее замкнутое множество, содержащее $M$, и потому оно может быть записано в виде $\bigcap\limits_{M \subset F} F$.

\vspace{0.5cm}

\addcontentsline{toc}{subsection}{Теорема 1.2}
\begin{Theorem} \label{thm:thm_1_2}
$(X, \rho)$ -- \MS. $\{G_{\alpha}\}$ -- семейство открытых множеств, $\{F_{\alpha}\}$ -- семейство замкнутых множеств, $\alpha \in A$. Тогда
\begin{center}
    \begin{tabular}{ |c|c| }
        \hline
        $\bigcup G_{\alpha}$ -- открытое & $\bigcap F_{\alpha}$ -- замкнутое \\[0.2cm]
        \hline
        $\bigcap\limits_{k=1}^n G_k$ -- открытое & $\bigcup\limits_{k=1}^n F_k$ -- замкнутое \\
        \hline
    \end{tabular}
    \end{center}
\end{Theorem}
\begin{Proof}
    Докажем верхнюю строчку таблица, нижняя оставляется в качестве упражнения. Нам пригодятся две формулы де Моргана (шотландский математик, 1806-1871): 
    $C \bigcup B_{\alpha} = \bigcap CB_{\alpha}$ и $C \bigcap B_{\alpha} = \bigcup CB_{\alpha}$, где $C$ обозначает операцию дополнения.
    \begin{enumerate}
        \item Для любой точки $x \in \bigcup G_{\alpha}$ можно выбрать $\exists \alpha_0:\, x \in G_{\alpha_0}$. Так как $G_{\alpha_0}$ открыто, то $\exists B(x) \subset G_{\alpha_0}$. Значит, $B(x) \subset \bigcup G_{\alpha}$.
        \item Обозначим $G_{\alpha} \defeq CF_{\alpha}$. Тогда воспользуемся формулой де Моргана: $\bigcap F_{\alpha} = \bigcap CG_{\alpha} = C \bigcup G_{\alpha}$. Так как $\bigcap G_{\alpha}$ открыто, то по Т.1.1 $\, C \bigcup G_{\alpha}$ замкнуто.
    \end{enumerate}
\end{Proof}

\addcontentsline{toc}{subsection}{Теорема 1.3}
\begin{Theorem} \label{thm:thm_1_3}
    $X$ -- \MS. Тогда верны следующие утверждения:
    \begin{enumerate}
        \item $B(x, r)$ -- открытое множество
        \item $\Int M$ -- открытое множество
        \item $\overline{M}$ -- замкнутое множество
        \item $\overline{B}(x, r)$ -- замкнутое множество
    \end{enumerate}
\end{Theorem}
\begin{Proof}
    \begin{enumerate}
        \item Для любой точки $y \in B(x, r)$ имеем оценку расстояния от центра шара до неё: $\rho(x, y) = r - \varepsilon$. Выберем окрестность точки $y$: $B(y, \frac{\varepsilon}{2})$.
        Теперь рассмотрим точки из этой окрестности. Для любой $z \in B(y, \frac{\varepsilon}{2})$: $\rho(z, x) \leq \rho(z, y) + \rho(y, x) = r - \frac{\varepsilon}{2} < r$. Значит, вся выбранная окрестность 
        точки $y$ лежит в $B(x, r)$, то есть $B(x, r)$ открыт.
        
        \item Простое замечание: $M_1 \subset M_2 \Rightarrow \Int M_1 \subset \Int M_2$. Перейдём к доказательству пункта. $\forall x \in \Int M \Rightarrow \exists B(x) \subset M \Rightarrow \Int B(x) \subset \Int M $. 
        А раз $B(x)$ открыто, то и $\Int M$ открыто.
    \end{enumerate}
    Доказательство оставшихся пунктов теоремы остаётся в качестве упражнения.
\end{Proof}

\vspace{0.5cm}

Интересно сравнить два объекта: $\overline{B}(x, r)$ и $\overline{B(x, r)}$. Всегда ли они равны? Оказывается, в данном случае опыт просто устроенных пространств не соответствует общему случаю,
эти два объекта не обязаны быть равными. Контрпримером служит пространство $(X, \rho)$, где $|X| > 2$ и $\rho(x, y) = \mathbb{I}[x=y]$.

\vspace{0.5cm}

Докажем полезный в дальнейшем факт о локальной топологии точчки прикосновения.

\begin{Statement}
    $x$ -- точка прикосновения множества $M \Longleftrightarrow \forall G(x): \, G(x) \cap M \neq \emptyset$, где $G(x)$ -- открытая окрестность точки $x$.
\end{Statement}
\begin{Proof}
    $\Rightarrow)$ Рассмотрим произвольную открытую окрестность точки $x$. Тогда $\exists B(x, r) \subset G$. По изначальному предположению: $B(x, r) \cap M \neq \emptyset$. Значит, $G \cap M \neq \emptyset$.
    \\
    $\Leftarrow)$ Очевидно.
\end{Proof}